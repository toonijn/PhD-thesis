% !TeX root = chapter1_introduction.tex
% !TeX root = thesis.tex
\ifdefined\UtilIncluded
  \renewcommand{\startchapter}[1]{}
  \renewcommand{\stopchapter}{}
\else

\newcommand{\startchapter}[1]{\begin{document}\setcounter{chapter}{#1}\addtocounter{chapter}{-1}}
\newcommand{\stopchapter}{\end{document}}


\documentclass[a4paper]{book}
\usepackage[utf8]{inputenc}

\usepackage{amsfonts,amsmath, amsthm, amssymb}
\usepackage{xspace}
\usepackage[hidelinks,bookmarks,pdfusetitle]{hyperref}
\usepackage{listings}
\usepackage[pdftex]{graphicx}
\usepackage{bm}
\usepackage[english]{babel}
\usepackage{caption}
\usepackage{subcaption}
\usepackage[usenames,dvipsnames]{xcolor}
\usepackage{physics}
\usepackage{multicol}
\usepackage{xstring}
\usepackage{pythonhighlight}
\usepackage{parskip}
\usepackage{thmtools}
\usepackage{relsize}
\usepackage{bookmark}
\usepackage{lmodern}
\usepackage{ifthen}
\usepackage{biblatex}
\usepackage{csquotes}

\addbibresource{references.bib}

\newtheorem{theorem}{Theorem}
\newtheorem{lemma}[theorem]{Lemma}
\newtheorem{corollary}[theorem]{Corollary}

\DeclareRobustCommand{\oneD}{{1{\relsize{-1}D}}\xspace}
\DeclareRobustCommand{\twoD}{{2{\relsize{-1}D}}\xspace}
\DeclareRobustCommand{\threeD}{{3{\relsize{-1}D}}\xspace}
\DeclareRobustCommand{\cpp}{{{C\nolinebreak[4]\hspace{-.05em}\raisebox{.4ex}{\relsize{-3}\textbf{++}}}\xspace}}
\pdfstringdefDisableCommands{%
    \def\cpp{C++}%
    \def\oneD{1D}%
    \def\twoD{2D}%
    \def\threeD{3D}%
}

\newcommand{\longchapter}[2][]{%
    \chapter[#2]{#2}%
    \ifthenelse{\equal{#1}{}}{}{\chaptermark{#1}}}

\fi
\gdef\UtilIncluded{}


\startchapter{1}

\chapter{Introduction to differential equations}

Differential equations recall in most mathematicians many feelings. Some only shiver by the mere idea of them, some of my colleagues in the more algebraic, (finite) geometric or discrete fields come to mind. Others, myself including, rejoice at the thought of studying them. 

The knowledge and experience about differential equations varies wildly between, even the best of, mathematicians. Some have only had one, maybe two, introductory courses, others have studied them their whole careers. To not mislead, I'll expand on this last sentence a bit more. There are very few mathematicians that can say they have studied `differential equations'. Neither can I, I have studied \emph{a} differential equation. Maybe two, if you count generously.



\section{History}


\subsection{Formalism}

\subsection{Advent of computing}

\subsection{Modern times}

\section{Ordinary differential equations}

\section{Partial differential equations}


\subsection{Eigenvalue-eigenfunction problems}

\subsection{Linear self-adjoint elliptic operators}

\begin{theorem}[Eigenvalues are sorted and increasing]
\end{theorem}

\begin{theorem}[Eigenfunctions are orthonormal]
\end{theorem}

\begin{theorem}[Courant's nodal domain theorem]
\end{theorem}

\stopchapter
