% !TeX root = chapter1_introduction.tex
% !TeX root = thesis.tex
\ifdefined\UtilIncluded
  \renewcommand{\startchapter}[1]{}
  \renewcommand{\stopchapter}{}
  \renewcommand{\undefinedlabel}[2]{}
\else

\newcommand{\startchapter}[1]{\begin{document}\setcounter{chapter}{#1}\addtocounter{chapter}{-1}}
\newcommand{\stopchapter}{\printbibliography[title=Bibliography,heading=bibintoc]\end{document}}


\documentclass{book}
\usepackage[utf8]{inputenc}


\usepackage{geometry}
\geometry{
  papersize={170mm,240mm},
}

\usepackage{amsfonts,amsmath, amsthm, amssymb, mathtools}
\usepackage{xspace}
\usepackage[hidelinks,bookmarks,pdfusetitle]{hyperref}
\usepackage{listings}
\usepackage[pdftex]{graphicx}
\usepackage{bm}
\usepackage[english]{babel}
\usepackage{caption}
\usepackage{subcaption}
\usepackage[usenames,dvipsnames]{xcolor}
\usepackage{physics}
\usepackage{multicol}
\usepackage{xstring}
\usepackage{pythonhighlight}
\usepackage{parskip}
\usepackage{thmtools}
\usepackage{relsize}
\usepackage{bookmark}
\usepackage{lmodern}
\usepackage{ifthen}
\usepackage{biblatex}
\usepackage{microtype}
\usepackage{csquotes}
\usepackage{numprint}
\usepackage{mleftright}
\npthousandsep{{\ifmmode\mskip2mu\else\hskip0.2em\fi}}
\npdecimalsign{.}

\addbibresource{references.bib}

\newtheorem{theorem}{Theorem}[chapter]
\newtheorem{lemma}[theorem]{Lemma}
\newtheorem{corollary}[theorem]{Corollary}
\newtheorem{definition}[theorem]{Definition}

\DeclareRobustCommand{\oneD}{{1{\relsize{-1}D}}\xspace}
\DeclareRobustCommand{\twoD}{{2{\relsize{-1}D}}\xspace}
\DeclareRobustCommand{\threeD}{{3{\relsize{-1}D}}\xspace}
\DeclareRobustCommand{\cpp}{{{C\nolinebreak[4]\hspace{-.05em}\raisebox{.4ex}{\relsize{-3}\textbf{++}}}\xspace}}
\pdfstringdefDisableCommands{%
  \def\cpp{C++}%
  \def\oneD{1D}%
  \def\twoD{2D}%
  \def\threeD{3D}%
}

\newcommand{\longchapter}[2][]{%
  \chapter[#2]{#2}%
  \ifthenelse{\equal{#1}{}}{}{\chaptermark{#1}}}

\newcommand{\NN}{\mathbb{N}}
\newcommand{\ZZ}{\mathbb{Z}}
\newcommand{\QQ}{\mathbb{Q}}
\newcommand{\QQbar}{\overline{\mathbb{Q}}}
\newcommand{\RR}{\mathbb{R}}
\newcommand{\CC}{\mathbb{C}}

\newcommand{\Eigen}{\texttt{Eigen}}

\newcommand{\sage}{\texttt{sage}\xspace}

\newcommand{\hamiltonian}{\mathcal{H}}

\newcommand{\transposesign}{\intercal}
\newcommand{\transpose}[1]{{#1}^\transposesign}
\newcommand{\adjointsign}{\text{H}}
\newcommand{\adjoint}[1]{{#1}^\adjointsign}

\newcommand{\xmin}{{x_{\text{min}}}}
\newcommand{\xmax}{{x_{\text{max}}}}
\newcommand{\ymin}{{y_{\text{min}}}}
\newcommand{\ymax}{{y_{\text{max}}}}

\newcommand{\Cbottom}{\vb{C}_\text{bottom}}
\newcommand{\Ctop}{\vb{C}_\text{top}}
\newcommand{\ubottom}{\vb{u}_\text{bottom}}
\newcommand{\utop}{\vb{u}_\text{top}}

\DeclareMathOperator{\diag}{diag}
\DeclareMathOperator{\tridiag}{tridiag}
\DeclareMathOperator{\eigs}{eigs}
\DeclareMathOperator*{\argmin}{arg\,min}
\DeclareMathOperator{\Ai}{Ai}
\DeclareMathOperator{\Bi}{Bi}
\DeclareMathOperator{\OO}{\mathcal{O}}

% https://tex.stackexchange.com/a/18192/163747
\makeatletter
\newcommand{\undefinedlabel}[2]{%
  \protected@write \@auxout {}{\string \newlabel {#1}{{#2}{\thepage}{#2}{#1}{}} }%
  \hypertarget{#1}{}
}
\makeatother

\fi
\gdef\UtilIncluded{}


\startchapter{1}

\chapter{Introduction to differential equations}

Differential equations recall in most mathematicians many feelings. Some only shiver by the mere idea of them, some of my colleagues in the more algebraic, (finite) geometric or discrete fields come to mind. Others, myself including, rejoice at the thought of studying them.

The knowledge and experience about differential equations varies wildly between, even the best of, mathematicians. Some have only had one, maybe two, introductory courses, others have studied them their whole careers. To not mislead, I'll expand on this last sentence a bit more. There are very few mathematicians that can say they have studied `differential equations'. Neither can I, I have studied \emph{a} differential equation. Maybe two, if you count generously.

Before diving into the required mathematical background, let us take a step back. Mathematics does not live in a vacuum. Modern ideas have grown out of a very rich history, with many influences from the scientific questions of the time. In this introduction we will take the time to appreciate this history, and discover how differential equations have been developed.

\section{History}

In this section, we will walk through an abridged version of the history of differential equations. Before talking about functions, it is important to talk about what makes up these functions.

A good starting point is to take a look at number. Counting things is universal and timeless, as such the natural numbers $\NN$ are a jumping point to all the other numbers. In modern mathematics, the next logical step is to introduce the integers $\ZZ$. But historically, negative numbers, and the concept of zero as a number, were only widely accepted throughout the Middle Ages, by Indian, Chinese and Arabic mathematicians.

In historic texts we found already references to (positive) fractions, as early as Ancient Egypt, around 1000 BC. So after $\NN$ the next numbers that were `discovered', were the positive rational numbers $\QQ^+$. And soon thereafter some irrational numbers were found. The most prominent example is probably $\sqrt{2}$, as the diagonal of the unit square. But it is a bit too generous to say that this was the discovery of the real numbers. It is more accurate to say that there was a notion of algebraic numbers $\QQbar$. These are the numbers that are solutions of polynomial equations, like $x^2 = 2$. But only around the year 900, the Egyptian mathematician Abū Kāmil Shujā ibn Aslam started to accept these solutions as numbers in and of itself.

The mathematical invention that may have had the most impact in our daily lives, may be also unnoticed by many: the Hindu-Arabic numeral system. This is a way of writing down numbers, invented between the first and fourth centuries. Most numeral systems were not made for large numbers or were cumbersome to work with. Only the best of mathematicians could do computation in those systems, especially multiplication was difficult. The Hindu-Arabic numeral system solved this by being positional based. This means that the last digits is the one's place, the digit before that ten's, then hundred's, and so on. This allows for a compact way to write down large numbers, that still allows easy computation through arithmetic manipulations. Our modern numeral systems is a direct descendant of the Hindu-Arabic numeral system. Compare for example the Roman numeral \uppercase\expandafter{\romannumeral 3846\relax} to our modern equivalent $3846$. And to help illustrate the point, try squaring both numbers.

From the sixteenth century onwards mathematics in Europe started to flourish. For our purposes, the next stride towards the real numbers was made by Simon Stevin from Bruges. He created a way to write numbers as a decimal expansion. It was he who introduced the concept of 'digits after the decimal point'. In this same century the complex numbers were discovered. And later in 1637, it was René Descartes who coined the terms `real' and `imaginary' numbers.

But it still took more than 200 years more before we would get a first formal mathematical definition of the real numbers $\RR$. It was the work of the great logicians of the late 19th century, spearheaded by Georg Cantor. He provided in 1874 a formal logical construction of the real numbers. It may seem surprising that a rigorous definition of the real numbers came so late in the rich history of mathematics. But, in practice, the development of calculus, and the theory of functions, did not really require strict formal definitions to advance.


\subsection{Calculus}




\subsection{Advent of computing}

\subsection{Modern times}

\section{Ordinary differential equations}

\section{Partial differential equations}


\subsection{Eigenvalue-eigenfunction problems}

\subsection{Linear self-adjoint elliptic operators}\label{sec:c1_selfadjoint}



\begin{theorem}[Eigenvalues are sorted and increasing]
\end{theorem}

\begin{theorem}[Eigenfunctions are orthonormal]
\end{theorem}

\begin{theorem}[Courant's nodal domain theorem]
\end{theorem}

\stopchapter
