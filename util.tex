% !TeX root = thesis.tex
\ifdefined\UtilIncluded
  \renewcommand{\startchapter}[1]{}
  \renewcommand{\stopchapter}{}
  \renewcommand{\undefinedlabel}[2]{}
\else

\newcommand{\startchapter}[1]{\begin{document}\setcounter{chapter}{#1}\addtocounter{chapter}{-1}}
\newcommand{\stopchapter}{\begin{uCJK}\printbibliography[title=Bibliography,heading=bibintoc]\end{uCJK}\end{document}}


\documentclass{book}
\usepackage[utf8]{inputenc}


\usepackage{geometry}
\geometry{
  papersize={170mm,240mm},
}

\usepackage{amsfonts,amsmath, amsthm, amssymb, mathtools}
\usepackage{xspace}
\usepackage[hidelinks,bookmarks,pdfusetitle,bookmarksopen,bookmarksdepth=3,bookmarksnumbered]{hyperref}
\usepackage{listings}
\usepackage[pdftex]{graphicx}
\usepackage{bm}
\usepackage[english]{babel}
\usepackage{caption}
\usepackage{subcaption}
\usepackage[usenames,dvipsnames]{xcolor}
\usepackage{physics}
\usepackage{multicol}
\usepackage{xstring}
\usepackage{parskip}
\usepackage{thmtools}
\usepackage{relsize}
\usepackage{bookmark}
\usepackage{lmodern}
\usepackage{ifthen}
\usepackage{biblatex}
\usepackage{microtype}
\usepackage{mleftright}
\usepackage{algpseudocode}
\usepackage{algorithm}
\usepackage{tikz}
\usepackage{booktabs}
\usepackage{gensymb}
\usepackage{etoolbox}
\usepackage{enumitem} % More abut lists
\usepackage{nicefrac}
\usepackage[outputdir=build]{minted} % Code highlighting
\usepackage{csquotes}
\usepackage{bxcjkjatype} % Japanese in bibliography
\usepackage{fancyhdr}
\usepackage{microtype} % Better interword spacing?
\pagestyle{fancy}
\fancyhf{}
\fancyhead[LE,RO]{\thepage}
\fancyhead[CO]{\leftmark}
\fancyhead[CE]{\rightmark}



\makeatletter
\newcommand{\padnum}[2]{%
  \ifnum#1>1 \ifnum#2<10 0\fi
  \ifnum#1>2 \ifnum#2<100 0\fi
  \ifnum#1>3 \ifnum#2<1000 0\fi
  \ifnum#1>4 \ifnum#2<10000 0\fi
  \ifnum#1>5 \ifnum#2<100000 0\fi
  \ifnum#1>6 \ifnum#2<1000000 0\fi
  \ifnum#1>7 \ifnum#2<10000000 0\fi
  \ifnum#1>8 \ifnum#2<100000000 0\fi
  \ifnum#1>9 \ifnum#2<1000000000 0\fi
  \fi\fi\fi\fi\fi\fi\fi\fi\fi
  \expandafter\@firstofone\expandafter{\number#2}%
}
\makeatother

\ifdefined\mainthesisfile
% \newcommand{\flipbookpath}[1]{flipbook/pdf/#1.pdf}
\newcounter{oddpagecounter}
\setcounter{oddpagecounter}{8}
\fancyfoot[OR]{%
    \stepcounter{oddpagecounter}%
    % \includegraphics[width=2cm]{flipbook/pdf/\padnum{4}{\value{oddpagecounter}}.pdf}
}
%\includegraphics[width=\paperwidth]{\flipbookpath{\padzeroes[2]{\decimal{oddpagecounter}}}}

%\usepackage{flipbook}
%\usepackage{fancyhdr}
%\fancyfoot[RO]{                           % Flipbook at right foot in odd pages
%\setlength\unitlength{1cm}                % Specify the units
%  \begin{picture}(0,0)                    % New picture
%    \put(-0.2,-2){                        % Position of the picture
%      \fbImageF{./flipbook/pdf/}{png}{width=2cm} % Insert numbered picture in increasing order. Directory: ./Fig/Flipbook. Prefix of all pictures: image_ (in fact image_1, image_2, ...). Extension of the pictures: png. 
%     }
% \end{picture}
%}
\fi


% nice code colors
\usemintedstyle{ugent}
\setminted{linenos=true, frame=leftline, framesep=3mm, numbersep=1mm}

% format numbers
\usepackage{numprint}
\npthousandsep{{\ifmmode\mskip2mu\else\hskip0.2em\fi}}
\npdecimalsign{.}

% define ugent colors
\usepackage{xcolor}
\definecolor{ugent_wit}{HTML}{FFFFFF}
\definecolor{ugent_zwart}{HTML}{000000}
\definecolor{ugent_blauw}{HTML}{1E64C8}
\definecolor{ugent_geel}{HTML}{FFD200}
\definecolor{ugent_oranje}{HTML}{F1A42B}
\definecolor{ugent_rood}{HTML}{DC4E28}
\definecolor{ugent_aqua}{HTML}{2D8CA8}
\definecolor{ugent_roze}{HTML}{E85E71}
\definecolor{ugent_hemelsblauw}{HTML}{8BBEE8}
\definecolor{ugent_lichtgroen}{HTML}{AEB050}
\definecolor{ugent_paars}{HTML}{825491}
\definecolor{ugent_warmoranje}{HTML}{FB7E3A}
\definecolor{ugent_turquoise}{HTML}{27ABAD}
\definecolor{ugent_lichtpaars}{HTML}{BE5190}
\definecolor{ugent_groen}{HTML}{71A860}

% do not break footnotes
\interfootnotelinepenalty=10000


\addbibresource{references.bib}

\newtheorem{theorem}{Theorem}[chapter]
\newtheorem{lemma}[theorem]{Lemma}
\newtheorem{corollary}[theorem]{Corollary}
\newtheorem{definition}[theorem]{Definition}
\newtheorem{example}[theorem]{Example}

\DeclareRobustCommand{\oneD}{{1{\relsize{-1}D}}\xspace}
\DeclareRobustCommand{\twoD}{{2{\relsize{-1}D}}\xspace}
\DeclareRobustCommand{\threeD}{{3{\relsize{-1}D}}\xspace}
\DeclareRobustCommand{\cpp}{{{C\nolinebreak[4]\hspace{-.05em}\raisebox{.4ex}{\relsize{-3}\textbf{++}}}\xspace}}
\pdfstringdefDisableCommands{%
  \def\cpp{C++}%
  \def\oneD{1D}%
  \def\twoD{2D}%
  \def\threeD{3D}%
}

\newcommand{\longchapter}[2][]{%
  \chapter[#2]{#2}%
  \ifthenelse{\equal{#1}{}}{}{\chaptermark{#1}}}

\newcommand{\todo}[1]{\textcolor{red}{To do: #1}}

\newcommand{\NN}{\mathbb{N}}
\newcommand{\ZZ}{\mathbb{Z}}
\newcommand{\QQ}{\mathbb{Q}}
\newcommand{\QQbar}{\overline{\mathbb{Q}}}
\newcommand{\RR}{\mathbb{R}}
\newcommand{\CC}{\mathbb{C}}
\newcommand{\Nplus}{\NN^{+}}

\newcommand{\Eigen}{\texttt{Eigen}}
\newcommand{\slepc}{\texttt{SLEPc}}
\newcommand{\spectra}{\texttt{spectra}}
\newcommand{\scipy}{\texttt{SciPy}}
\newcommand{\arpack}{\texttt{ARPACK}}
\newcommand{\superlu}{\texttt{SuperLU}}

\newcommand{\sage}{\texttt{sage}}
\newcommand{\maple}{\texttt{maple}}
\newcommand{\matlab}{\textsc{matlab}}
\newcommand{\mathematica}{\texttt{mathematica}}

\newcommand{\catch}{\texttt{catch}}
\newcommand{\lpython}{\texttt{python}}
\newcommand{\javascript}{\texttt{javascript}}
\newcommand{\java}{\texttt{java}}
\newcommand{\csharp}{\texttt{C\#}}
\newcommand{\rust}{\texttt{rust}}
\newcommand{\fortran}{\texttt{Fortran}}
\newcommand{\matscs}{\texttt{MatSCS}}
\newcommand{\pyslise}{\texttt{Pyslise}}
\newcommand{\pyslisetd}{\texttt{Pyslise2D}}
\newcommand{\lilix}{\texttt{LILIX}}
\newcommand{\pybind}{\texttt{pybind}}
\newcommand{\matslise}[1]{\texttt{Matslise\ifstrempty{#1}{}{\,\mbox{{#1}\hspace{-1pt}.\hspace{-1pt}0}}}}

\newcommand{\hamiltonian}{\mathcal{H}}

\newcommand{\transposesign}{\intercal}
\newcommand{\transpose}[1]{{#1}^\transposesign}
\newcommand{\adjointsign}{\text{H}}
\newcommand{\adjoint}[1]{{#1}^\adjointsign}

\newcommand{\xmin}{{x_{\text{min}}}}
\newcommand{\xmax}{{x_{\text{max}}}}
\newcommand{\ymin}{{y_{\text{min}}}}
\newcommand{\ymax}{{y_{\text{max}}}}

\newcommand{\Cbottom}{\vb{C}_\text{bottom}}
\newcommand{\Ctop}{\vb{C}_\text{top}}
\newcommand{\ubottom}{\vb{u}_\text{bottom}}
\newcommand{\utop}{\vb{u}_\text{top}}
\newcommand{\dOmega}{\partial\Omega}

\newcommand{\dirdiam}{\operatorname{dirdiam}}
\DeclareMathOperator{\diam}{diam}
\DeclareMathOperator{\meas}{meas}

\DeclareMathOperator{\diag}{diag}
\DeclareMathOperator{\tridiag}{tridiag}
\DeclareMathOperator{\spectrum}{\sigma}
\DeclareMathOperator*{\argmin}{arg\,min}
\DeclareMathOperator{\Ai}{Ai}
\DeclareMathOperator{\Bi}{Bi}
\DeclareMathOperator{\OO}{\mathcal{O}}
\DeclareMathOperator{\opnull}{null}
\DeclareMathOperator{\opran}{ran}
\DeclareMathOperator{\oprank}{rank}
\DeclareMathOperator{\opspan}{span}

% https://tex.stackexchange.com/a/18192/163747
\makeatletter
\newcommand{\undefinedlabel}[2]{%
  \protected@write \@auxout {}{\string \newlabel {#1}{{#2}{\thepage}{#2}{#1}{}} }%
  \hypertarget{#1}{}
}
\makeatother

\setcounter{secnumdepth}{3}

\fi
\gdef\UtilIncluded{}

