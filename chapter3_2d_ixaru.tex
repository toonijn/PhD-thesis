% !TeX root = chapter3_2d_ixaru.tex
% !TeX root = thesis.tex
\ifdefined\UtilIncluded
  \renewcommand{\startchapter}[1]{}
  \renewcommand{\stopchapter}{}
  \renewcommand{\undefinedlabel}[2]{}
\else

\newcommand{\startchapter}[1]{\begin{document}\setcounter{chapter}{#1}\addtocounter{chapter}{-1}}
\newcommand{\stopchapter}{\printbibliography[title=Bibliography,heading=bibintoc]\end{document}}


\documentclass{book}
\usepackage[utf8]{inputenc}


\usepackage{geometry}
\geometry{
  papersize={170mm,240mm},
}

\usepackage{amsfonts,amsmath, amsthm, amssymb, mathtools}
\usepackage{xspace}
\usepackage[hidelinks,bookmarks,pdfusetitle]{hyperref}
\usepackage{listings}
\usepackage[pdftex]{graphicx}
\usepackage{bm}
\usepackage[english]{babel}
\usepackage{caption}
\usepackage{subcaption}
\usepackage[usenames,dvipsnames]{xcolor}
\usepackage{physics}
\usepackage{multicol}
\usepackage{xstring}
\usepackage{pythonhighlight}
\usepackage{parskip}
\usepackage{thmtools}
\usepackage{relsize}
\usepackage{bookmark}
\usepackage{lmodern}
\usepackage{ifthen}
\usepackage{biblatex}
\usepackage{microtype}
\usepackage{csquotes}
\usepackage{numprint}
\usepackage{mleftright}
\npthousandsep{{\ifmmode\mskip2mu\else\hskip0.2em\fi}}
\npdecimalsign{.}

\addbibresource{references.bib}

\newtheorem{theorem}{Theorem}[chapter]
\newtheorem{lemma}[theorem]{Lemma}
\newtheorem{corollary}[theorem]{Corollary}
\newtheorem{definition}[theorem]{Definition}

\DeclareRobustCommand{\oneD}{{1{\relsize{-1}D}}\xspace}
\DeclareRobustCommand{\twoD}{{2{\relsize{-1}D}}\xspace}
\DeclareRobustCommand{\threeD}{{3{\relsize{-1}D}}\xspace}
\DeclareRobustCommand{\cpp}{{{C\nolinebreak[4]\hspace{-.05em}\raisebox{.4ex}{\relsize{-3}\textbf{++}}}\xspace}}
\pdfstringdefDisableCommands{%
  \def\cpp{C++}%
  \def\oneD{1D}%
  \def\twoD{2D}%
  \def\threeD{3D}%
}

\newcommand{\longchapter}[2][]{%
  \chapter[#2]{#2}%
  \ifthenelse{\equal{#1}{}}{}{\chaptermark{#1}}}

\newcommand{\NN}{\mathbb{N}}
\newcommand{\ZZ}{\mathbb{Z}}
\newcommand{\QQ}{\mathbb{Q}}
\newcommand{\QQbar}{\overline{\mathbb{Q}}}
\newcommand{\RR}{\mathbb{R}}
\newcommand{\CC}{\mathbb{C}}

\newcommand{\Eigen}{\texttt{Eigen}}

\newcommand{\sage}{\texttt{sage}\xspace}

\newcommand{\hamiltonian}{\mathcal{H}}

\newcommand{\transposesign}{\intercal}
\newcommand{\transpose}[1]{{#1}^\transposesign}
\newcommand{\adjointsign}{\text{H}}
\newcommand{\adjoint}[1]{{#1}^\adjointsign}

\newcommand{\xmin}{{x_{\text{min}}}}
\newcommand{\xmax}{{x_{\text{max}}}}
\newcommand{\ymin}{{y_{\text{min}}}}
\newcommand{\ymax}{{y_{\text{max}}}}

\newcommand{\Cbottom}{\vb{C}_\text{bottom}}
\newcommand{\Ctop}{\vb{C}_\text{top}}
\newcommand{\ubottom}{\vb{u}_\text{bottom}}
\newcommand{\utop}{\vb{u}_\text{top}}

\DeclareMathOperator{\diag}{diag}
\DeclareMathOperator{\tridiag}{tridiag}
\DeclareMathOperator{\eigs}{eigs}
\DeclareMathOperator*{\argmin}{arg\,min}
\DeclareMathOperator{\Ai}{Ai}
\DeclareMathOperator{\Bi}{Bi}
\DeclareMathOperator{\OO}{\mathcal{O}}

% https://tex.stackexchange.com/a/18192/163747
\makeatletter
\newcommand{\undefinedlabel}[2]{%
  \protected@write \@auxout {}{\string \newlabel {#1}{{#2}{\thepage}{#2}{#1}{}} }%
  \hypertarget{#1}{}
}
\makeatother

\fi
\gdef\UtilIncluded{}


\startchapter{3}

\undefinedlabel{sec:intro_selfadjoing}{1.\~}

\longchapter[A shooting method]{A shooting method for the \twoD time-independent Schrödinger equation}

There are many general methods for partial differential equations. Each method has its own benefits and disadvantages. As a rule of thumb one can say that a method which is very general and widely applicable, will be less efficient, less accurate or both then a method which is specifically tuned for the problem at hand. With that in mind, there is a real advantage to gain when investing time and research into a highly tuned optimized method for a specific problem.

In this, and the next, chapter we will study the two-dimensional time-independent Schrödinger equations
\begin{equation}\label{equ:schrodinger_2d}
    -\nabla\psi(x, y) + V(x, y) \psi(x, y) = \lambda \psi(x, y)
\end{equation}
on the domain $\Omega = [\xmin, \xmax]\times[\ymin, \ymax]$. We will only consider homogenous Dirichlet boundary conditions, this means $\forall (x, y) \in \delta\Omega : \psi(x, y) = 0$. The function $V: \RR^2 \to \RR$ is called the potential function. This potential, together with the domain, defines the Schrödinger problem. When \emph{solving} the time-independent Schrödinger equation, one is searching for values for $\lambda$ such that a function $\psi(x, y)$  exists such they together satisfy the equation \eqref{equ:schrodinger_2d}. Such a value $\lambda$ is called an \emph{eigenvalue} with the corresponding \emph{eigenfunction} $\psi(x, y)$.

Equation \eqref{equ:schrodinger_2d} can also be interpreted as finding the eigenvalues of the \emph{Hamiltonian} $H$:
$$
    H := -\nabla + V(x, y)\text{.}
$$
\begin{theorem}
    The Hamiltonian operator is self-adjoint. And some more adjectives.
\end{theorem}

Proving the Hamiltonian is a linear self-adjoint operator allows us to employ the theory as described in section \ref{sec:intro_selfadjoing}.

In \cite{ixaru_new_2010} a new method is proposed to approximate solutions of the two-dimensional time-independent Schrödinger equation.

\section{The method}

\section{Our improvements}

\subsection{}

\stopchapter
