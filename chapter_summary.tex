% !TeX root = chapter_software.tex
% !TeX root = thesis.tex
\ifdefined\UtilIncluded
  \renewcommand{\startchapter}[1]{}
  \renewcommand{\stopchapter}{}
\else

\newcommand{\startchapter}[1]{\begin{document}\setcounter{chapter}{#1}\addtocounter{chapter}{-1}}
\newcommand{\stopchapter}{\end{document}}


\documentclass[a4paper]{book}
\usepackage[utf8]{inputenc}

\usepackage{amsfonts,amsmath, amsthm, amssymb}
\usepackage{xspace}
\usepackage[hidelinks,bookmarks,pdfusetitle]{hyperref}
\usepackage{listings}
\usepackage[pdftex]{graphicx}
\usepackage{bm}
\usepackage[english]{babel}
\usepackage{caption}
\usepackage{subcaption}
\usepackage[usenames,dvipsnames]{xcolor}
\usepackage{physics}
\usepackage{multicol}
\usepackage{xstring}
\usepackage{pythonhighlight}
\usepackage{parskip}
\usepackage{thmtools}
\usepackage{relsize}
\usepackage{bookmark}
\usepackage{lmodern}
\usepackage{ifthen}
\usepackage{biblatex}
\usepackage{csquotes}

\addbibresource{references.bib}

\newtheorem{theorem}{Theorem}
\newtheorem{lemma}[theorem]{Lemma}
\newtheorem{corollary}[theorem]{Corollary}

\DeclareRobustCommand{\oneD}{{1{\relsize{-1}D}}\xspace}
\DeclareRobustCommand{\twoD}{{2{\relsize{-1}D}}\xspace}
\DeclareRobustCommand{\threeD}{{3{\relsize{-1}D}}\xspace}
\DeclareRobustCommand{\cpp}{{{C\nolinebreak[4]\hspace{-.05em}\raisebox{.4ex}{\relsize{-3}\textbf{++}}}\xspace}}
\pdfstringdefDisableCommands{%
    \def\cpp{C++}%
    \def\oneD{1D}%
    \def\twoD{2D}%
    \def\threeD{3D}%
}

\newcommand{\longchapter}[2][]{%
    \chapter[#2]{#2}%
    \ifthenelse{\equal{#1}{}}{}{\chaptermark{#1}}}

\fi
\gdef\UtilIncluded{}


\startchapter{0}

\undefinedlabel{cha:c2}{2}
\undefinedlabel{cha:c3}{3}
\undefinedlabel{cha:c4}{4}

\chapter*{Summary}
\addcontentsline{toc}{chapter}{Summary}


\section*{English summary}
\addcontentsline{toc}{section}{English summary}

An $n$-dimensional time-independent Schrödinger equation is a linear second order partial differential equation given by
\begin{equation}\label{equ:sum_en_schrodinger}
-\nabla^2 \psi(\vb{x}) + V(\vb{x}) \psi(\vb{x}) = E\vb{x}\text{.}
\end{equation}
In this expression, $V$ is a provided potential functions defined on the domain $\Omega \subseteq \RR^n$. When solving this equation, the goal is to find all functions $\psi : \Omega \to \RR$ and values $E \in \RR$ that satisfy the Schrödinger equation~\eqref{equ:sum_en_schrodinger} in combination with provided boundary conditions. In such a solution $E$ is called the eigenvalue, with corresponding eigenfunction $\psi$.

For the one-dimensional case, Schrödinger equations are a special case of Sturm--Liouville equations:
\begin{equation}\label{equ:sum_en_sl}
    -(p(x), y'(x))' + q(x) y(x) = \lambda w(x) y(x) \text{.}
\end{equation}
In this equation $p(x)$, $q(x)$ and $w(x)$ are given functions on a connected domain $[a, b] \subseteq \RR$. Here, $\lambda$ is the unknown eigenvalue with corresponding eigenfunction $y$.

In chapter \ref{cha:c2}, an existing numerical method for approximating eigenvalues and eigenfunctions of \eqref{equ:sum_en_sl} is studied. Here regular Sturm--Liouville problems on $[a, b]$ are considered with homogeneous Robin boundary conditions:
$$
\alpha_a y(a) + \beta_a p(a) y'(a) = 0 \text{ and } \alpha_b y(b) + \beta_b p(b) y'(b) = 0\text{.}
$$
This constant-perturbation method is able to reach extremely accurate results, for small as well as high eigenvalues.

\todo{Why is ours better?}

\todo{Two-dimensional problems.}

% LTeX: language=nl-BE
\section*{Nederlandse samenvatting}
\addcontentsline{toc}{section}{Nederlandse samenvatting}


\stopchapter
