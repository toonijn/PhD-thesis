% !TeX root = chapter_summary.tex
% !TeX root = thesis.tex
\ifdefined\UtilIncluded
  \renewcommand{\startchapter}[1]{}
  \renewcommand{\stopchapter}{}
\else

\newcommand{\startchapter}[1]{\begin{document}\setcounter{chapter}{#1}\addtocounter{chapter}{-1}}
\newcommand{\stopchapter}{\end{document}}


\documentclass[a4paper]{book}
\usepackage[utf8]{inputenc}

\usepackage{amsfonts,amsmath, amsthm, amssymb}
\usepackage{xspace}
\usepackage[hidelinks,bookmarks,pdfusetitle]{hyperref}
\usepackage{listings}
\usepackage[pdftex]{graphicx}
\usepackage{bm}
\usepackage[english]{babel}
\usepackage{caption}
\usepackage{subcaption}
\usepackage[usenames,dvipsnames]{xcolor}
\usepackage{physics}
\usepackage{multicol}
\usepackage{xstring}
\usepackage{pythonhighlight}
\usepackage{parskip}
\usepackage{thmtools}
\usepackage{relsize}
\usepackage{bookmark}
\usepackage{lmodern}
\usepackage{ifthen}
\usepackage{biblatex}
\usepackage{csquotes}

\addbibresource{references.bib}

\newtheorem{theorem}{Theorem}
\newtheorem{lemma}[theorem]{Lemma}
\newtheorem{corollary}[theorem]{Corollary}

\DeclareRobustCommand{\oneD}{{1{\relsize{-1}D}}\xspace}
\DeclareRobustCommand{\twoD}{{2{\relsize{-1}D}}\xspace}
\DeclareRobustCommand{\threeD}{{3{\relsize{-1}D}}\xspace}
\DeclareRobustCommand{\cpp}{{{C\nolinebreak[4]\hspace{-.05em}\raisebox{.4ex}{\relsize{-3}\textbf{++}}}\xspace}}
\pdfstringdefDisableCommands{%
    \def\cpp{C++}%
    \def\oneD{1D}%
    \def\twoD{2D}%
    \def\threeD{3D}%
}

\newcommand{\longchapter}[2][]{%
    \chapter[#2]{#2}%
    \ifthenelse{\equal{#1}{}}{}{\chaptermark{#1}}}

\fi
\gdef\UtilIncluded{}


\startchapter{0}

\undefinedlabel{cha:c2}{2}
\undefinedlabel{cha:c3}{3}
\undefinedlabel{cha:c4}{4}

\chapter*{Summary}
\addcontentsline{toc}{chapter}{Summary}


\section*{English summary}
\addcontentsline{toc}{section}{English summary}

An $n$-dimensional time-independent Schrödinger equation is a linear second order partial differential equation given by
\begin{equation}\label{equ:sum_en_schrodinger}
-\nabla^2 \psi(\vb{x}) + V(\vb{x}) \psi(\vb{x}) = E\vb{x}\text{.}
\end{equation}
In this expression, $V$ is a provided potential function defined on the domain $\Omega \subseteq \RR^n$. When solving this equation, the goal is to find all functions $\psi : \Omega \to \RR$ and values $E \in \RR$ that satisfy the Schrödinger equation~\eqref{equ:sum_en_schrodinger} in combination with provided boundary conditions. In such a solution $E$ is called the eigenvalue, with corresponding eigenfunction $\psi$.

For the one-dimensional case, Schrödinger equations are a special case of Sturm--Liouville equations:
\begin{equation}\label{equ:sum_en_sl}
    -(p(x), y'(x))' + q(x) y(x) = \lambda w(x) y(x) \text{.}
\end{equation}
In this equation $p(x)$, $q(x)$ and $w(x)$ are given functions on a connected domain $[a, b] \subseteq \RR$. Here, $\lambda$ is the unknown eigenvalue with corresponding eigenfunction $y$.

In chapter~\ref{cha:c2}, the existing constant-perturbation method for approximating eigenvalues and eigenfunctions of~\eqref{equ:sum_en_sl} is studied. Here regular Sturm--Liouville problems on $[a, b]$ are considered with homogeneous Robin boundary conditions:
$$
\alpha_a y(a) + \beta_a p(a) y'(a) = 0 \text{ and } \alpha_b y(b) + \beta_b p(b) y'(b) = 0\text{.}
$$
This constant-perturbation method is able to reach extremely accurate results, for small as well as high eigenvalues. With this method, we build a new \cpp{}-implementation \matslise{3} accompanied by the \lpython{}-package \pyslise{}. The new implementation is up to $100$ times faster than \matslise{2} when calculating eigenvalues. For the evaluation of eigenfunctions, our program is even up to a $1000$ times faster. These speed-ups are possible, due to the development and calculation of new more complicated propagation formulae.

We also consider (generalized) periodic boundary conditions. This is the first time, as far as we can tell, a constant-perturbation method is used for these kinds of problems. The main challenge here is making sure \emph{all} requested eigenvalues are found.

In chapters~\ref{cha:c3} and~\ref{cha:c4}, two methods for the approximation of eigenvalues and eigenfunctions of the time-independent two-dimensional Schrödinger equation~\eqref{equ:sum_en_schrodinger}  are studied. The first method (from chapter~\ref{cha:c3}) is an existing method developed by Ixaru based upon constant-perturbation ideas. It is limited to Schrödinger problems on rectangular domains with homogeneous Dirichlet boundary conditions. Throughout that chapter, we make some improvements to this method. Most notably, we develop a technique to reliably determine the index of an eigenvalue. This enables us to ensure all requested eigenvalues are found.

The second method for approximating eigenvalues and eigenfunctions of time-independent two-dimensional Schrödinger equations with homogeneous Dirichlet boundary conditions on (possibly non-rectangular) domains can be found in chapter \ref{cha:c4}. Here, we are inspired by a simple method based upon a finite difference scheme. This leads us to develop Strands, a \cpp{}-program accompanied by a \lpython{}-package with the same name. By using collocation ideas with well-chosen basis functions (inspired by chapter~\ref{cha:c3}), this new method is able to reach much more accurate results, with a similar computational cost.

% LTeX: language=nl-BE
\section*{Nederlandse samenvatting}
\addcontentsline{toc}{section}{Nederlandse samenvatting}


\stopchapter
