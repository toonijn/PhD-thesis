% !TeX root = chapter_minted_test.tex
% !TeX root = thesis.tex
\ifdefined\UtilIncluded
  \renewcommand{\startchapter}[1]{}
  \renewcommand{\stopchapter}{}
\else

\newcommand{\startchapter}[1]{\begin{document}\setcounter{chapter}{#1}\addtocounter{chapter}{-1}}
\newcommand{\stopchapter}{\end{document}}


\documentclass[a4paper]{book}
\usepackage[utf8]{inputenc}

\usepackage{amsfonts,amsmath, amsthm, amssymb}
\usepackage{xspace}
\usepackage[hidelinks,bookmarks,pdfusetitle]{hyperref}
\usepackage{listings}
\usepackage[pdftex]{graphicx}
\usepackage{bm}
\usepackage[english]{babel}
\usepackage{caption}
\usepackage{subcaption}
\usepackage[usenames,dvipsnames]{xcolor}
\usepackage{physics}
\usepackage{multicol}
\usepackage{xstring}
\usepackage{pythonhighlight}
\usepackage{parskip}
\usepackage{thmtools}
\usepackage{relsize}
\usepackage{bookmark}
\usepackage{lmodern}
\usepackage{ifthen}
\usepackage{biblatex}
\usepackage{csquotes}

\addbibresource{references.bib}

\newtheorem{theorem}{Theorem}
\newtheorem{lemma}[theorem]{Lemma}
\newtheorem{corollary}[theorem]{Corollary}

\DeclareRobustCommand{\oneD}{{1{\relsize{-1}D}}\xspace}
\DeclareRobustCommand{\twoD}{{2{\relsize{-1}D}}\xspace}
\DeclareRobustCommand{\threeD}{{3{\relsize{-1}D}}\xspace}
\DeclareRobustCommand{\cpp}{{{C\nolinebreak[4]\hspace{-.05em}\raisebox{.4ex}{\relsize{-3}\textbf{++}}}\xspace}}
\pdfstringdefDisableCommands{%
    \def\cpp{C++}%
    \def\oneD{1D}%
    \def\twoD{2D}%
    \def\threeD{3D}%
}

\newcommand{\longchapter}[2][]{%
    \chapter[#2]{#2}%
    \ifthenelse{\equal{#1}{}}{}{\chaptermark{#1}}}

\fi
\gdef\UtilIncluded{}


\startchapter{1}

\subsection{Bash}

% \begin{noindent}
\begin{minted}{bash}
pip install strands
\end{minted}
% \end{noindent}

\subsection{Python}
% \begin{noindent}
\begin{minted}{python}
from strands import Schrodinger2D, Rectangle
import numpy as np
import matplotlib.pyplot as plt

xs = np.linspace(-9.5, 9.5, 200)
ys = np.linspace(-9.5, 9.5, 200)
X, Y = np.meshgrid(xs, ys)

# Test

schrodinger = Schrodinger2D(
    lambda x, y: x * x + y * y, Rectangle(-9.5, 9.5, -9.5, 9.5),
    gridSize=(40, 40), maxBasisSize=30)

for E, f in schrodinger.eigenfunctions(10):
    plt.pcolormesh(X, Y, f(X, Y))
    plt.show()
\end{minted}
% \end{noindent}

\subsection{C++}
% \begin{noindent}
\begin{minted}{c++}
#include <iostream>
#include <boost/format.hpp>
#include <boost/math/constants/constants.hpp>
#include <boost/multiprecision/float128.hpp>
#include <matslise/matslise.h>

using boost::math::constants::pi;
using boost::multiprecision::float128;

float128 mathieuPotential(float128 x) {
    // Test comment
    return 2 * cos(2 * x);
}

int main() {
    /*
        Block comment
    */
    matslise::Matslise<float128> problem(
            &mathieuPotential, 0, pi<float128>(), 1e-25q);

    auto boundary = matslise::Y<float128>::Dirichlet();
    auto eigs = problem.eigenvaluesByIndex(0, 7, boundary);
    for (auto [i, E]: eigs) {
        float128 error = problem.eigenvalueError(E, boundary, i);
        std::cout << boost::format(
                "Eigenvalue %1$d:%2$30.25f  (error: %3$.1e)")
                        % i % E % error << std::endl;
    }
    return 0;
}
\end{minted}
% \end{noindent}

\stopchapter
