% !TeX root = chapter4_2d_new.tex
% !TeX root = thesis.tex
\ifdefined\UtilIncluded
  \renewcommand{\startchapter}[1]{}
  \renewcommand{\stopchapter}{}
\else

\newcommand{\startchapter}[1]{\begin{document}\setcounter{chapter}{#1}\addtocounter{chapter}{-1}}
\newcommand{\stopchapter}{\end{document}}


\documentclass[a4paper]{book}
\usepackage[utf8]{inputenc}

\usepackage{amsfonts,amsmath, amsthm, amssymb}
\usepackage{xspace}
\usepackage[hidelinks,bookmarks,pdfusetitle]{hyperref}
\usepackage{listings}
\usepackage[pdftex]{graphicx}
\usepackage{bm}
\usepackage[english]{babel}
\usepackage{caption}
\usepackage{subcaption}
\usepackage[usenames,dvipsnames]{xcolor}
\usepackage{physics}
\usepackage{multicol}
\usepackage{xstring}
\usepackage{pythonhighlight}
\usepackage{parskip}
\usepackage{thmtools}
\usepackage{relsize}
\usepackage{bookmark}
\usepackage{lmodern}
\usepackage{ifthen}
\usepackage{biblatex}
\usepackage{csquotes}

\addbibresource{references.bib}

\newtheorem{theorem}{Theorem}
\newtheorem{lemma}[theorem]{Lemma}
\newtheorem{corollary}[theorem]{Corollary}

\DeclareRobustCommand{\oneD}{{1{\relsize{-1}D}}\xspace}
\DeclareRobustCommand{\twoD}{{2{\relsize{-1}D}}\xspace}
\DeclareRobustCommand{\threeD}{{3{\relsize{-1}D}}\xspace}
\DeclareRobustCommand{\cpp}{{{C\nolinebreak[4]\hspace{-.05em}\raisebox{.4ex}{\relsize{-3}\textbf{++}}}\xspace}}
\pdfstringdefDisableCommands{%
    \def\cpp{C++}%
    \def\oneD{1D}%
    \def\twoD{2D}%
    \def\threeD{3D}%
}

\newcommand{\longchapter}[2][]{%
    \chapter[#2]{#2}%
    \ifthenelse{\equal{#1}{}}{}{\chaptermark{#1}}}

\fi
\gdef\UtilIncluded{}


\startchapter{4}
\undefinedlabel{cha:c2}{2}
\undefinedlabel{cha:c3}{3}

\longchapter[New methods]{New methods for the \twoD time-independent Schrödinger equation}\label{cha:c4}

In chapter \ref{cha:c2} we have studied the constant perturbation methods. We have seen a brief history about these CP-methods, as well as, a thorough overview about how these methods can be implemented. The numerical examples illustrated the benefits and demonstrated the accuracy of the studied techniques.

Chapter \ref{cha:c3} was dedicated to the treatment of a recent method to solve the time-independent two-dimensional Schrödinger equation. This method aims to use the strengths of the constant perturbation methods for higher dimensional problems. This new method is promising, and we developed many improvements upon the original idea.

One of the unique powers of the CP-methods was their ability to not only compute low eigenvalues accurately, but even increase accuracy for higher eigenvalues. This is one of the few, if not the only, method which has this very desirable property for Sturm-Liouville problems. For two dimensions, this property did not translate cleanly. The method described in chapter \ref{cha:c3} tries to capture this by considering solutions of a one-dimensional Schrödinger problem in the $x$-direction. And a method for a coupled system of Schrödinger equations has been used in the $y$-direction. In theory this method is capable of computing any eigenvalue. In practice, this is not the case. Along the $x$-direction an eigenfunction is represented as a linear combination of basis functions. This basis is well-chosen, such that only a few number of functions are able to capture the true eigenfunctions sufficiently. Yet, as computers are finite, this basis has to be this as well. As eigenfunctions corresponding to higher eigenvalues will become more and more oscillatory, the chosen finite basis will no longer be able to express all necessary details.

That the basis is finite, thus, limits the accuracy for higher eigenvalues. Which negates one of the strongest benefits of the employed CP-methods. As such, we believe that the utopian method for the more-dimensional time-independent Schrödinger equation, is one which does not decrease in accuracy, as the higher eigenvalues are requested. Just like the CP-methods are for the one-dimensional case. Developing such a method will, most likely, require new and very complicated formulae.

For clarity, we did not develop such a perfect method. But in the last few years I've played with the idea... In section \ref{c4:sec_utopy} we will present some ideas for such a method. Some of which, someone, somewhere, may find inspirational.

More realistically, during our research into the method of Ixaru, we have found other research, also focussing upon the time-independent Schrödinger problem. These ideas and methods have inspired us to develop our own technique. The new methods we propose try to fix or mitigate some issues present in the other methods.

    {\color{red} To do: list other methods}

\section{Inspiration}

After careful implementation and thorough testing of Ixaru's method from chapter \ref{cha:c3}, I could browse through the literature with renewed appreciation for the Schrödinger problem. When researching that method, many obstacles and challenges arose. It was an interesting task to balance computation time, symbolic formulas and numerical accuracy. So with this in mind, I came across \cite{wang_new_2009} by Wang and Shao.

In this article they proposed a new kind of discretization scheme for solving two-dimensional time-independent Schrödinger equations. Before studying the details of the method, I like to review the numerical results. They tested their algorithm on two potential functions. Firstly, the harmonic oscillator: Schrödinger equations with this potential function are not extremely difficult, but it captures some of the challenges that numerical algorithms may face, while still having symbolic solutions for the eigenvalues. Secondly, they provided results for the Hénon-Heiles potential. Here, the exact solutions are not known, but reliable approximations exist. For both potentials their results were quite impressive. The reached high accuracy, for a significant number of eigenvalues, without using excessive computation time.

But the thing that strikes me most about \cite{wang_new_2009} is the simplicity of their method. They constructed formulas for a direct discrete approximation on a grid. Finding these kinds of results within the literature is disheartening and inspirational at the same time. At first glance they were able to reach higher accuracy with a simpler method than Ixaru's work \cite{ixaru_new_2010} and our improvements \cite{baeyens_improvements_2022} to it. Before throwing away all our work, and declaring \cite{wang_new_2009} to be superior, let us analyze it ourselves.

\subsection{A finite difference scheme}

In \cite{wang_new_2009} they discretize space with an equidistant grid. With this, they approximate the second partial derivative of a function $\psi(x, y)$ with, what they call, a new scheme:
$$
    {\pdv[2]{}{x}}\psi(x, y) = -\frac{1}{h^2}\sum_{i=-N_q}{N_q} c_i \psi(x + ih, y)\text{.}
$$
In this expression $h$ is the $x$-step of the equidistant grid. An analogous formula for the second partial derivative of $\psi(x, y)$ with respect to $y$ is proposed, with $y$-step $\eta$:
$$
    {\pdv[2]{}{y}}\psi(x, y) = -\frac{1}{\eta^2}\sum_{i=-N_q}{N_q} c_i \psi(x, y + i\eta)\text{.}
$$

To determine the value of $c_i$, the authors propose to use the Taylor series expansion of
$$
    \left({\pdv[2]{}{x}} + {\pdv[2]{}{y}}\right)\psi(x, y) = -\frac{1}{h^2} \sum_{i=-N_q}^{N_q} c_i \left(\psi(x+ih, y) + \left(\frac{h}{\eta}\right)^2\psi(x, y+i\eta)\right)\text{.}
$$
In \cite{wang_new_2009} the coefficients $c_i$ for $N_q = 3, 4, 5, 6$ are provided. If we restrict the Schrödinger equation to the domain $[-R_x, R_x] \times [-R_y, R_y]$ the authors propose to assume all eigenfunctions will be zero outside this region. This way, their formulae, for a fixed $N_q$ can still be used, even for grid points close to the edge of the domain.

Using these formulae the problem now approximates to a simple, albeit very large, square matrix eigenvalue problem. Many classical, well tested,  solvers exist for this kind of problem, they have used algorithms provided by Mathematica.

Before reviewing, and recalculating their numerical results, I want to take the time to thoroughly analyze their formulae. Using an equidistant grid and coefficients which optimize for the Taylor series of the function in question, is oddly reminiscent of finite difference approximations. To confirm this similarity we have calculated the central finite difference approximations of the second order derivative.

The following table contains the first few central symmetric finite differences.
\begin{center}
    \begin{tabular}{r|ccccccccccccc}
Order & $c_0$ & $c_{-1, 1}$ & $c_{-2, 2}$ & $c_{-3, 3}$ & $c_{-4, 4}$ & $c_{-5, 5}$ & $c_{-6, 6}$\\[2pt]\hline
2 & $-2$ & $1$ & & & & &\\[4pt]
4 & $- \frac{5}{2}$ & $\frac{4}{3}$ & $- \frac{1}{12}$ & & & &\\[4pt]
6 & $- \frac{49}{18}$ & $\frac{3}{2}$ & $- \frac{3}{20}$ & $\frac{1}{90}$ & & &\\[4pt]
8 & $- \frac{205}{72}$ & $\frac{8}{5}$ & $- \frac{1}{5}$ & $\frac{8}{315}$ & $- \frac{1}{560}$ & &\\[4pt]
10 & $- \frac{5269}{1800}$ & $\frac{5}{3}$ & $- \frac{5}{21}$ & $\frac{5}{126}$ & $- \frac{5}{1008}$ & $\frac{1}{3150}$ &\\[4pt]
12 & $- \frac{5369}{1800}$ & $\frac{12}{7}$ & $- \frac{15}{56}$ & $\frac{10}{189}$ & $- \frac{1}{112}$ & $\frac{2}{1925}$ & $- \frac{1}{16632}$
\end{tabular}

\end{center}

As one sees, these give exactly the same coefficients as in \cite{wang_new_2009}. It is quite disingenuous to call this a new scheme. The study of finite difference can be traced back to, among others, Newton. One of the first English works was by Boole \cite{boole_calculus_1860} in 1860, with many more textbooks to follow \cite{thomson_calculus_1933,jordan_calculus_1965}. Even the symbolic construction of high order approximations has already been studied and tabulated as early as 1967 \cite{ballester_construction_1967,keller_symbolic_1978,fornberg_generation_1988}.


Besides this remark, \cite{wang_new_2009} is very valuable as an application of very high orders of these well-known formulae to the two-dimensional time-independent Schrödinger equation. There numerical results are still valid and nonetheless impressive.

\subsubsection{Numerical experiments with high order finite difference approximations}

{\color{red}To do: experiments}

\subsubsection{Possible improvements when applying finite differences}

{\color{red}To do

    \begin{enumerate}
        \item Do not assume zero outside domain
        \item High orders loose sparsity
    \end{enumerate}

}

\subsection{A semi-discrete method}\label{sec:c4_semi_discrete}

\section{A woven interpolation method}

In contrast to the previous methods, we propose a technique which is not necessarily restricted to cases where the domain is only a rectangle. So consider a finite domain $\Omega \subseteq \RR$ on which we are searching eigenvalues $E \in \RR$ and eigenfunctions $\psi: \Omega \to \RR$ such that for a given potential $V : \Omega \to \RR$ the following holds
\begin{equation}\label{equ:c4_schrodinger_equation_new_method}
    -\nabla^2 \psi + V(x, y) \psi = E \psi\text{.}
\end{equation}
Still, we impose homogenous Dirichlet boundary conditions, thus, $\psi(x, y) = 0$ for $(x, y) \in \Omega$.

The main idea underlying this method is that we want to represent the eigenfunctions $\psi$ efficiently. A fully discretized method may represent eigenfunctions as it's values on certain grid points. Our first attempt to develop a more continuous approximation of the eigenfunction used ideas from chapter \ref{cha:c3} and led to the creation of section \ref{sec:c4_semi_discrete}. Here we have chosen to approximate the eigenfunction as a linear combination of well-chosen basis functions on parallel lines throughout the domain. This led to a continuous approximation along one direction and a discrete approximation along the other direction of the domain.

\begin{figure}
    \begin{center}
        \includegraphics[width=.8\linewidth]{img/chapter4/the_method_grid.pdf}
        \caption{\label{fig:woven_method_grid} The grid}
    \end{center}
\end{figure}

In this new method we will bring this continuous approximation to both directions of the domain. For this, we place a grid over the domain $\Omega$, as can be seen in figure \ref{fig:woven_method_grid}. This grid need not be equidistant or square. When developing this new method we strived to allow for maximal flexibility, by avoiding as many restrictions on $\Omega$ as possible. This has as consequence that, because $\Omega$ does not have to be a rectangle, the number of intersections per grid line is not necessarily constant. Upon formulating and implementing this new method, this varying number of intersections is one example of some difficulties that arise by explicitly allowing more flexibility.

After placing the grid on $\Omega$, the next step is to approximate the unknown eigenfunction $\psi$ as a linear combination of basis functions on each of the grid lines. On the vertical line $x = x_i$, we denote these basis functions as $\beta_k^{(x_i)}(y)$, for each value of $k$. This yields the expression
$$
    \psi(x_i, y) = \sum_{k=0}^\infty c_k^{(x_i)} \beta_k^{(x_i)}(y) \text{.}
$$
For horizontal lines $y = y_j$, the basis functions $\beta_k^{(y_j)}(x)$ yield a similar expression
$$
    \psi(x, y_j) = \sum_{k=0}^\infty c_k^{(y_j)} \beta_k^{(y_j)}(x) \text{.}
$$

To ensure $\psi(x, y)$ is uniquely defined in each point in $\Omega$ we have to require that in each intersection point $(x_i, y_j)$, $\psi(x_i, y_j)$ has only one solution:
\begin{equation}\label{equ:c4_new_method_pre_matrix_equality}
    \psi(x_i, y_j) = \sum_{k=0}^\infty c_k^{(x_i)} \beta_k^{(x_i)}(y_j) = \sum_{k=0}^\infty c_k^{(y_j)} \beta_k^{(y_j)}(x_i)\text{.}
\end{equation}

Before deciding on which basis functions we should use, let us consider the Schrödinger equation \eqref{equ:c4_schrodinger_equation_new_method} on each intersection point $(x_i, y_j)$ with this new representation of $\psi$:
$$
    -\sum_{k=0}^\infty c_k^{(x_i)} \beta''^{(x_i)}_k(y_j) - \sum_{k=0}^\infty c_k^{(y_j)} \beta''^{(y_j)}_k(x_i) + (V(x_i, y_j) - E) \psi(x_i, y_j) = 0\text{.}
$$

This last formula suggest to chose $\beta_k^{(x_i)}$ and $\beta_k^{(y_j)}$, such that its second derivative contains, in a certain sense, $V(x_i, y_j)$. Together with the idea from chapter \ref{cha:c3} to use a one-dimensional Schrödinger equation, this leads us to propose $\beta_k^{(x_i)}$ and $\beta_k^{(y_j)}$ to be the ordered eigenfunctions which satisfy the one dimensional Schrödinger equation
$$
    -\beta_k''^{(x_i)}(y) + \frac{V(x_i, y)}{2}\beta_k^{(x_i)}(y) = \lambda_k^{(x_i)} \beta_k^{(x_i)}(y)
$$
with homogenous Dirichlet boundary conditions. The domain for this one-dimensional problem is the intersection of the vertical line $x = x_i$ and the two-dimensional domain $\Omega$. Similarly, for the horizontal line $y = y_j$, we propose $\beta_k^{(y_j)}(x)$ to be the eigenfunctions of
$$
    -\beta_k''^{(y_j)}(x) + \frac{V(x, y_j)}{2}\beta_k^{(y_j)}(x) = \lambda_k^{(y_j)} \beta_k^{(y_j)}(x)
$$
with homogenous Dirichlet boundary conditions, and as domain the intersection of $y = y_j$ and $\Omega$.

By choosing half the original potential in each of the approximations, in each intersection $(x_i, y_j)$, equation \eqref{equ:c4_schrodinger_equation_new_method} simplifies, and $V(x, y)$ disappears:
\begin{equation}\label{equ:c4_new_method_pre_matrix}
    \sum_{k=0}^\infty \lambda_k^{(x_i)} c_k^{(x_i)} \beta^{(x_i)}_k(y_j) + \sum_{k=0}^\infty \lambda_k^{(y_j)} c_k^{(y_j)} \beta_k^{(y_j)}(x_i) = E \psi(x_i, y_j) \text{.}
\end{equation}

Notice that in this expression only $E$ (the eigenvalue) and $c_k^{x_i}$ and $c_k^{y_j}$ are unknown, as $\psi(x_i, y_j)$ depends linearly on $c_k^{x_i}$ and $c_k^{y_j}$. So, expression \eqref{equ:c4_new_method_pre_matrix} is, in fact, a linear problem.

Before writing this as a matrix-problem, we first have to limit the extent of the sum. It is, of course, impossible to implement these formulas while the function bases are still infinite. Therefor, we limit the basis on the line $x = x_i$ to the first $K_{x_i}$ functions and to the first $K_{y_j}$ functions on the line $y = y_j$. Notice that the size of a basis on each line, should not be greater than the number of intersections on that line. If the basis size is to larger, the system \eqref{equ:c4_new_method_pre_matrix_equality} would be underdetermined.

With these finite sums equations \eqref{equ:c4_new_method_pre_matrix_equality} and \eqref{equ:c4_new_method_pre_matrix} become, on the intersection $(x_i, y_j)$:
\begin{align}
     & \sum_{k=0}^{K_{x_i}-1} \lambda_k^{(x_i)} c_k^{(x_i)} \beta^{(x_i)}_k(y_j) + \sum_{k=0}^{K_{y_j}-1} \lambda_k^{(y_j)} c_k^{(y_j)} \beta_k^{(y_j)}(x_i)\nonumber                                  \\
     & \qquad\qquad\qquad     = \sum_{k=0}^{K_{x_i}-1} c_k^{(x_i)} \beta_k^{(x_i)}(y_j) = \sum_{k=0}^{K_{y_j}-1} c_k^{(y_j)} \beta_k^{(y_j)}(x_i) \text{.}\label{equ:c4_new_method_pre_matrix_unified}
\end{align}
Introducing appropriate vectors and matrices will allow is to translate this into a matrix-problem. The unknowns will be summarized into the two vectors $\vb{c_x}$ and $\vb{c_y}$, with sizes $n_x := \sum_i K_{x_i}$ and $n_y := \sum_j K_{y_j}$ respectively:
\begin{align*}
    \vb{c_x}            & = \transpose{\begin{pmatrix} c_0^{(x_0)} & c_1^{(x_0)} & \dots & c_{K_{x_0}-1}^{(x_0)} & c_0^{(x_1)} & c_1^{(x_1)} & \dots \end{pmatrix}}         \\
    \text{and }\vb{c_y} & = \transpose{\begin{pmatrix} c_0^{(y_0)} & c_1^{(y_0)} & \dots & c_{K_{y_0}-1}^{(y_0)} & c_0^{(y_1)} & c_1^{(y_1)} & \dots \end{pmatrix}}\text{.}
\end{align*}

Furthermore, we introduce the $n_x \times n_x$ diagonal matrix $\vb{\Lambda_x}$ and the $n_y \times n_y$ diagonal matrix $\vb{\Lambda_y}$ which contain the eigenvalues of the one-dimensional Schrödinger problems used to define the basis functions:
\begin{align*}
    \vb{\Lambda_x}            & = \diag\begin{pmatrix} \lambda_0^{(x_0)} & \lambda_1^{(x_0)} & \dots & \lambda_{K_{x_0}-1}^{(x_0)} & \lambda_0^{(x_1)} & \lambda_1^{(x_1)} & \dots \end{pmatrix}         \\
    \text{and }\vb{\Lambda_y} & = \diag\begin{pmatrix} \lambda_0^{(y_0)} & \lambda_1^{(y_0)} & \dots & \lambda_{K_{y_0}-1}^{(y_0)} & \lambda_0^{(y_1)} & \lambda_1^{(y_1)} & \dots \end{pmatrix}\text{.} \\
\end{align*}

\begin{figure}
    \begin{center}
        \includegraphics[width=\textwidth]{img/chapter4/new_method_beta.pdf}
        \caption{The non-zero entries from the matrices $\vb{B_x}$ and $\vb{B_y}$ from equation \eqref{equ:c4_new_method_beta_definition} are visualized. These were calculated on the problem from section \ref{sec:c4_new_method_ixaru} with a 5 by 5 internal grid and 4 basis functions per line. In practice, these matrices are much larger.}
        \label{fig:c4_new_method_beta}
    \end{center}
\end{figure}

Lastly we define the matrices which contain the values of the basis functions in each of the grid points. For this let us define $m$ as the total number of intersection points. Now, we define the $m \times n_x$ matrix $\vb{B_x}$ and the $m \times n_y$ matrix $\vb{B_y}$. Each row $r$ of these matrices correspond to an intersection $(x_i, y_j)$. Matrix $\vb{B_x}$ is mostly zero, except, on row $r$, for the values in columns from $o_x := \sum_{i' < i} K_{x_{i'}}$ to before $o_x + K_i$. Analogous, is $\vb{B_y}$ mostly zero on row $r$, except for the values in the columns from $o_y := \sum_{j' < j} K_{y_{j'}}$ to before $o_y + K_{y_j}$. These non-zero entries are now given as:
\begin{align}
    \left(\vb{B_x}\right)_{r, o_x + k} & =  \beta_{k}^{(x_i)}(y_j) \text{ for $k \in \{0, 1, \dots, K_{x_i} - 1\}$ }   \nonumber                               \\
    \text{and }
    \left(\vb{B_y}\right)_{r, o_y + k} & =  \beta_{k}^{(y_j)}(x_i) \text{ for $k \in \{0, 1, \dots, K_{y_i} - 1\}$.} \label{equ:c4_new_method_beta_definition}
\end{align}
To aid an intuitive understanding of the $\vb{B_x}$ and $\vb{B_y}$ matrices, figure \ref{fig:c4_new_method_beta} provides a schematic view of them, calculated from a small numerical example.


As promised, these definitions allow us to rewrite the system \eqref{equ:c4_new_method_pre_matrix_unified} with $m$ equations more compactly as:
$$
    \vb{B_x} \vb{\Lambda_x}\vb{c_x} + \vb{B_y} \vb{\Lambda_y}\vb{c_y} = E \vb{B_x} \vb{c_x} = E \vb{B_y} \vb{c_y}\text{.}
$$
This formulation is not yet reminiscent of any classical linear algebra problem. One of the unfamiliar parts of this expression is the fact that there are two vectors of unknowns, another unfamiliar part is that this are, in fact, two different problems.
\begin{equation}
    \begin{cases}
        \vb{B_x} \vb{\Lambda_x}\vb{c_x} + \vb{B_y} \vb{\Lambda_y}\vb{c_y} = E \vb{B_x} \vb{c_x} \\
        \vb{B_x} \vb{c_x} = \vb{B_y} \vb{c_y}
    \end{cases} \label{equ:c4_new_method_first_matrix_problem}
\end{equation}

There are a few strategies to further translate this problem into a form for which efficiently implemented and well-studied algorithms exist. Ideally, the solution would take the sparsity of the involved matrices into account.

\subsection{Solving the matrix approximation}

As a first strategy we researched how this problem can be directly transformed into a known type of problem.

\subsubsection{By direct transformation into an eigenvalue problem}

We tackle equation \eqref{equ:c4_new_method_first_matrix_problem} by rewriting it as
$$
    \begin{cases}
        \vb{B_x} \vb{\Lambda_x}\vb{c_x} + \vb{B_y} \vb{\Lambda_y}\vb{c_y} = E \vb{B_x} \vb{c_x} \\
        \vb{B_x} \vb{\Lambda_x}\vb{c_x} + \vb{B_y} \vb{\Lambda_y}\vb{c_y} = E \vb{B_y} \vb{c_y} \\
    \end{cases}\text{.}
$$
This system can be seen as the following generalized rectangular eigenvalue problem:
\begin{equation}\label{equ:c4_new_method_alternative_system}
    \begin{pmatrix}
        \vb{B_x} \vb{\Lambda_x} & \vb{B_y} \vb{\Lambda_y} \\
        \vb{B_x} \vb{\Lambda_x} & \vb{B_y} \vb{\Lambda_y}
    \end{pmatrix} \begin{pmatrix}
        \vb{c_x} \\ \vb{c_y}
    \end{pmatrix} = E \begin{pmatrix}
        \vb{B_x} & \vb{0} \\ \vb{0} & \vb{B_y}
    \end{pmatrix} \begin{pmatrix}
        \vb{c_x} \\ \vb{c_y}
    \end{pmatrix}\text{.}
\end{equation}

At first glance this may seem to enable us to solve the problem elegantly. But, there are a few issues apparent with this translation. One of the most visible problems is that we have translated this into a matrix problem which is twice as large in both rows and columns. On the other hand, one can argue, the matrices are clearly sparse. The matrices $\vb{B_x}$ and $\vb{B_y}$ are sparse indeed, but the problem is that there are few algorithms available that are able to solve a generalized rectangular eigenvalue problem, let alone a sparse one.

But a lack of sparse implementations, does not deter us from trying out some numerical experiments. In this first experiment we will have to fall back to algorithms working on dense matrices. Of course, the runtime will suffer, but from a numerical point of view the results will still be valuable.

Consider the Schrödinger equation with the harmonic oscillator potential:
$$
    -\nabla^2\psi(x, y) + \left(x^2 + y^2\right) \psi(x, y) = E \psi(x, y)
$$
on the domain $[-10, 10] \times [-10, 10]$ with homogenous Dirichlet boundary conditions. We apply the described method on a grid with 30 lines in each direction and 15 basis functions per line. This yields $900\times 420$ matrices $\vb{B_x}$ and $\vb{B_y}$, and the rectangular problem from \eqref{equ:c4_new_method_alternative_system} has \numprint{1800} equations and \numprint{840} variables. As this eigenvalue problem is heavily overdetermined, we have to considered that solutions will only be accurate in a least squares sense. Much research is already dedicated to solving this kind of problem, as such we will, for now, use the first method from \cite{hua_svd_1991} to find solutions. Later on, we will explore the world of generalized rectangular eigenvalue problems.

If we write equation \eqref{equ:c4_new_method_alternative_system} symbolically as $\vb{A} \vb{c} = E \vb{D} \vb{c}$, and the truncated singular value decomposition of $\vb{D}$ as $\vb{D} = \vb{U_D} \vb{\Sigma_D} \vb{V_D^\adjointsign}$, then any solution of \eqref{equ:c4_new_method_alternative_system} will also be a solution of the generalized square eigenvalue problem
$$
    \vb{U_D^\adjointsign} \vb{A} \vb{V_D} \vb{v} = E \vb{\Sigma_D} \vb{v} \text{,}
$$
with $\vb{c} = \vb{V_D} \vb{v}$. Applying this method to the numeric problem yields 840 eigenvalues. Firstly, it is important to remark that the resulting values are elements of $\CC$. But, in this case, the imaginary part of all values lies between \numprint{-1.36e-14} and \numprint{1.36e-14}. So all values may be considered real. The lowest few values are given:
$$
    \underbrace{
        \numprint{-3.75e-14} \quad \dots \quad \numprint{5.59e-14}
    }_\text{196 values close to zero} \quad \numprint{2.00} \quad \numprint{3.46} \quad \numprint{3.46} \quad \numprint{4.00} \quad \numprint{4.00} \quad \dots
$$

This can be more compactly summarized when repeated values are indicated by a subscript:
$$
    \numprint{0.00}_{196} \quad \numprint{2.00} \quad \numprint{3.46}_2 \quad \numprint{4.00} \quad \numprint{4.89}_2 \quad \numprint{6.00}_3 \quad \numprint{6.12}_2 \quad \dots \quad \numprint{7.73}_2 \quad \numprint{8.00}_4\quad \dots
$$

One immediately notices the many returned zero values. This does not have to be surprising, in fact, the matrix on the left-hand side of \eqref{equ:c4_new_method_alternative_system} may not be of full rank. But these zeros indicate another problem. Namely, that for these values it is not at all guaranteed that $\vb{B_x} \vb{c_x} = \vb{B_y} \vb{c_y}$.

Remember that the true eigenvalues of the harmonic oscillator are 2, 4, 4, 6, 6, 6, 8, 8, 8, 8, \dots. And, we notice that these true values can indeed be found in our solutions. But also many other values are present. Because the method from \cite{hua_svd_1991} for solving generalized rectangular eigenvalue problems may return more solutions than the original problems has, we still have to filter some out. One way to only end up with true eigenvalues is by substituting them back into the original problem \eqref{equ:c4_new_method_first_matrix_problem} and verifying the residuals of both equations:
$$
    r_1 = \left\| \vb{B_x} \vb{\Lambda_x} \vb{c_x} + \vb{B_y} \vb{\Lambda_y} \vb{c_y} - \frac{E}{2}\left(\vb{B_x}\vb{c_x} + \vb{B_y}\vb{c_y}\right) \right\| \text{ and } r_2=\left\| \vb{B_x} \vb{c_x} - \vb{B_y} \vb{c_y} \right\|\text{.}
$$

The found possible eigenvalue with the lowest residuals $r_1$ and $r_2$ are also true eigenvalues of the Schrödinger problem. When we sort all possibilities by their first residual $r_1$ we obtain the following table.

\begin{center}
    \begin{tabular}{r|rrrrr}
$E$ & \numprint{2.000000} & \numprint{4.000000} & \numprint{4.000000} & \numprint{6.000000} & \numprint{7.999998} \\\hline
$r_1$ & \numprint{1.5e-5} & \numprint{4.7e-4} & \numprint{4.7e-4} & \numprint{4.9e-4} & \numprint{5.0e-3} \\
$r_2$ & \numprint{1.2e-5} & \numprint{2.3e-4} & \numprint{2.3e-4} & \numprint{1.6e-4} & \numprint{1.2e-3}
\end{tabular}

\begin{tabular}{r|rrrrr}
$E$ & \numprint{7.999998} & \numprint{9.999997} & \numprint{9.999997} & \numprint{10.000006} & \numprint{8.000007} \\\hline
$r_1$ & \numprint{5.0e-3} & \numprint{6.6e-3} & \numprint{6.7e-3} & \numprint{1.0e-2} & \numprint{1.4e-2} \\
$r_2$ & \numprint{1.2e-3} & \numprint{1.3e-3} & \numprint{1.3e-3} & \numprint{2.1e-3} & \numprint{3.5e-3}
\end{tabular}

\begin{tabular}{r|rrrrr}
$E$ & \numprint{8.000007} & \numprint{13.999988} & \numprint{5.999974} & \numprint{5.999974} & \numprint{10.000151} \\\hline
$r_1$ & \numprint{1.4e-2} & \numprint{1.9e-2} & \numprint{3.8e-2} & \numprint{3.8e-2} & \numprint{4.8e-2} \\
$r_2$ & \numprint{3.5e-3} & \numprint{2.7e-3} & \numprint{1.3e-2} & \numprint{1.3e-2} & \numprint{9.6e-3}
\end{tabular}


\end{center}

Here, we see that when the first residual $r_1$ is low, the other is as well. Also, in the first few eigenvalues, only true solutions are present.

Upon studying this direct method to solve the system of equations \eqref{equ:c4_new_method_first_matrix_problem}, we have noticed some drawbacks. Firstly, the proposed system \eqref{equ:c4_new_method_alternative_system} is, in a certain sense, twice as large as the discrete problem we started from. Eigenvalue algorithms are at least cubic in complexity. So, this doubling in size implies an eightfold runtime penalty Secondly, and numerically interesting, we obtain many more `solutions' than the ones we are looking for. Each generalized eigenvalue, with its eigenvector, has to be computed and checked against the residuals. Many eigenvalues will be thrown away.

\subsubsection{By restriction to a null space}

One improvement we can make to the runtime is by not solving a system that is twice as large, but by solving two, smaller, systems. Furthermore, instead of ending up with to many eigenvalues and trying to filter out the wrong ones, we have devised a way to a priori already limit the number of solutions. The idea here is that, before solving an eigenvalue problem, we solve the second equation of \eqref{equ:c4_new_method_first_matrix_problem} and only take those solutions into account.

The first equation resembles a generalized rectangular eigenvalue problem, the second is a classical linear system. Let us only consider $n_x + n_y$ dimensional vectors $\vb{c} = \transpose{\begin{pmatrix}\transpose{\vb{c_x}} & \transpose{\vb{c_y}} \end{pmatrix}}$ which solve
$$
    \begin{pmatrix}\vb{B_x} & -\vb{B_y} \end{pmatrix} \vb{c} = \vb{0}\text{.}
$$
This allows us to unify $\vb{c_x}$ and $\vb{c_y}$, while ensuring $\vb{B_x} \vb{c_x} = \vb{B_y} \vb{c_y}$ is satisfied. To expand on this idea, we will write $\vb{c}$ to be an element of the right kernel of  $\begin{pmatrix}\vb{B_x} & -\vb{B_y} \end{pmatrix}$. For this define $\vb{Z}$ to be the $(n_x + n_y) \times z$-dimensional basis of this right kernel:
$$
    \begin{pmatrix}\vb{B_x} & -\vb{B_y} \end{pmatrix} \vb{Z} = \vb{0}\text{.}
$$
Note that numerically, it is impossible to consider the exact kernel. {\color{red} To do: the numerical treatment of the computation of this kernel.}

The vector $\vb{c}$ can now be written as a certain linear combination of columns of $\vb{Z}$:
$$
    \vb{c} = \begin{pmatrix}\vb{c_x} \\ \vb{c_y} \end{pmatrix} = \begin{pmatrix} \vb{Z_x} \\ \vb{Z_y} \end{pmatrix}  \vb{u} = \vb{Z} \vb{u} \text{.}
$$
Another benefit of considering $\vb{Z}\vb{u}$, besides only considering solutions of $\vb{B_x} \vb{c_x} = \vb{B_y} \vb{c_y}$, is that we unified the two vectors of unknowns $\vb{c_x}$ and $\vb{c_y}$ into one (much) smaller vector $\vb{u}$. This simplifies the two problems of equation \eqref{equ:c4_new_method_first_matrix_problem} into
$$
    \begin{pmatrix}
        \vb{B_x}\vb{\Lambda_x} & \vb{B_y}\vb{\Lambda_x}
    \end{pmatrix} \vb{Z} \vb{u} = E \vb{B_x} \vb{Z_x} \vb{u} = E \vb{B_y} \vb{Z_y} \vb{u} \text{.}
$$
For the right-hand side, by construction $\vb{B_x} \vb{Z_x} = \vb{B_y} \vb{Z_y}$. Now, this problem has become a generalized rectangular eigenvalue problem.


\subsubsection{Finding solutions of generalized rectangular eigenvalue problems}

{\color{red} To do: A small word about this kind of problem.}

\subsection{Computing eigenfunctions}

\subsection{Numerical experiments}

\subsubsection{Ixaru's potential}\label{sec:c4_new_method_ixaru}

\section{Some ideas for a utopian method}\label{c4:sec_utopy}


\stopchapter
